% !TEX root = ../fib_poly.tex

\subsection{Steenrod diagonals}

A \textit{cellular space} is a topological space with a CW structure and a \textit{cellular map} is a continuous map preserving skeleta.
The product of cellular spaces is naturally a cellular space.
For our purposes, two important examples of cellular spaces are the interval $\gcube$ with its usual CW structure, and the colimit $\gcube^\infty$ of the system of cellular maps $\gcube^0 \to \gcube^1 \to \gcube^2 \to \dotsb$ defined by inclusion into the subspace with last coordinate equal to $0$.

A cellular map $X \to X \times X$ is a \textit{cellular diagonal} if it is homotopic to the (non-cellular) diagonal $x \mapsto (x, x)$, and agrees with it on the $0$-cells of $X$.
Applying the functor of (cellular) chains $\gchains = \gchains(-\, ;\, \Z)$, a cellular diagonal $X \to X \times X$ makes $\gchains(X)$ into a coalgebra in $\Ch$, the category of chain complexes, which induces the cup product in the cohomology of $X$.

Unlike the diagonal of $X$, a cellular diagonal is not invariant under the transposition action of $\S_2$ on $X \times X$.
A \textit{Steenrod diagonal} is a cellular map $\Delta \colon \gcube^\infty \times X \to X \times X$ such that the restrictions $\Delta_i \colon \gcube^i \times X \to X \times X$ satisfy that $\Delta_0$ is a cellular diagonal and that, for $i > 0$, $\Delta_i$ is a homotopy between $\Delta_{i-1}$ and its transpose.
We refer to $\Delta_i$ as the \textit{Steenrod $i$-diagonal} of $\Delta$.
Applying the functor of chains, a Steenrod diagonal induces for $i \geq 0$ a degree $i$ linear map
\[
\gchains(\Delta_i) \colon \gchains(X) \to \gchains(X \times X) \xra{\cong} \gchains(X) \otimes \gchains(X)
\]
referred to as a \textit{cup-$i$ coproduct}.
These define the Steenrod squares on the mod~$2$ cohomology of $X$ by the formula
\[
Sq^k\big( [\alpha] \big) \defeq \big[ (\alpha \otimes \alpha)\gchains(\Delta_{k-\bars{\alpha}}) \big],
\]
where brackets denote the cohomology class represented by a cocycle and bars its (cohomological) degree.


