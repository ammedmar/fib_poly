% !TEX root = ../fib_poly.tex

We will be interested in the preceding construction when $P$ is the polytope of diagonals of some other polytope.

\begin{definition}
	Let $P\subset \R^n$ be a polytope.
	Its \emph{polytope of diagonals} $D_P$ is the fiber polytope $\Sigma_\pi(P\times P, P)$, where $\pi(x,y)\defeq (x+y)/2$.
\end{definition}

We observe that $P$ embeds into $P\times P$ via the set-theoretic diagonal $\Delta (x)=(x,x)$, which is a section of $\pi$.
The vertices of $D_P$ are associated with tight coherent sections, that are cellular approximations of the diagonal $\Delta$ \cite[Proposition 5]{MTTV19}, see also \cite[Proposition 1.1]{GLA21}.
Consider the diagonal embedding
\[
\begin{tikzcd}[row sep=0, column sep=small]
	\Delta \colon \R^n \arrow{r} & \R^n \times \R^n \\
	\phantom{\Delta \colon} z \arrow[r, |->] & (z,z).
\end{tikzcd}
\]
Denote by $\{e_j\}$ the standard basis of $\R^n$.
Then, $\{\Delta (e_j)\}$ is a basis of $\Ima \Delta$ and we have an isomorphism $\R^n \cong \Ima \Delta$.
A basis for the orthogonal complement $\Ima \Delta^{\bot}$ is $\{(e_j,-e_j)\}$, and we have an isomorphism $\R^n \cong \Ima \Delta^{\bot}$.
For any $z \in \R^n$, we have $\pi^{-1}(z)=\Delta(z)+\ker \pi$, and an affine isomorphism
\begin{equation} \label{eq:D_P-iso}
	\begin{matrix}
		\iota_z & : & \R^n  & \cong & \pi^{-1}(z) \\
		& & v  & \mapsto & \Delta(z)+(v,-v).
	\end{matrix}
\end{equation}
The polytope of diagonals $D_P$ has the same dimension as $P$, and can be seen in $\R^n$ via the isomorphism $\iota_0$.

\begin{remark}
	We can see this isomorphism as the discrete analogue of the isomorphism between the tangent bundle of a manifold $M$ and the normal bundle of the diagonal submanifold of $M\times M$.
\end{remark}

\begin{proposition}
	For any polytope $P$, the polytope of diagonals $D_P$ is centrally symmetric.
\end{proposition}

\begin{proof}
	Let $x=(y,-y)$ be a vertex of $D_P$.
	There is a linear form $\angles{-,v}$ which is maximized at $x$ over $D_P$.
	Its associated tight coherent section
	\begin{equation*}
		\begin{matrix}
			\gamma & : & P & \to & P \times P \\
			& & x  & \mapsto & (\gamma_1(z),\gamma_2(z))
		\end{matrix}
	\end{equation*}
	is given by maximum of $\angles{-,v}$ in each fiber $\pi^{-1}(z)$ of $\pi$. Then, the section
	\begin{equation*}
		\begin{matrix}
			\sigma_2\gamma & : & P & \to & P \times P \\
			& & x  & \mapsto & (\gamma_2(z),\gamma_1(z))
		\end{matrix}
	\end{equation*}
	is given by the minimum of $\angles{-,v}$ in each fiber, or equivalently the maximum of of $\angles{-,-v}$ in each fiber.
	Thus, this tight coherent section defines a vertex $-x=(-y,y)$ of $D_P$.
\end{proof}

In the sequel, we will be interested in a KV system over $D_P$.

\begin{definition}
	A \emph{Steenrod diagonal} on $P$ is the datum of a KV system on $D_P$ which possess the bot-top property with respect to $D_P$.
\end{definition}

We denote by $\rho_z P \defeq 2z-P$ the reflection of $P$ with respect to $z \in P$.

\begin{lemma}
	There is an affine isomorphism
	\begin{equation} \label{eq:iso-intersection}
		\begin{matrix}
			\rho & : & P\cap\rho_z P & \xra{\cong} & \pi^{-1}(z) \\
			& & x  & \mapsto & (x,2z-x)
		\end{matrix}
	\end{equation}
	for all $z \in P$.
\end{lemma}

\begin{proof}
	This is a straightforward observation.
	Equivalently, the isomorphism can be given by the assignment $z+t \mapsto (z+t,z-t)$.
\end{proof}

This lemma makes clear the pointwise description of the diagonal
\begin{equation*}
	\begin{matrix}
		\triangle_{(P,v)} & : & P & \to & P \times P \\
		& & z  & \mapsto & (\bm_v(P \cap \rho_z P),\tp_v(P \cap \rho_z P))
	\end{matrix}
\end{equation*}
in \cite[Definition 10]{MTTV19}, see also \cite[Proposition 1.15]{GLA21}.
Indeed, the tight coeherent section associated to a vertex in $D_P$ is given by the extremal vetices of all the fibers $\pi^{-1}(z), z \in P$ with respect to some functional $\angles{-,w}$.
Since $D_P \subset \ker \pi$, we can restrict ourselves to the vectors of the form $w=(v,-v) \in \ker \pi$ without loosing generality.
Under the isomorphism (\ref{eq:iso-intersection}) above, we see that the maximum (resp. minimum) of $(v,-v)$ over $\pi^{-1}(z)$ is exactly the top (resp. bot) element of $P\cap \rho_z P$ with respect to $v$.

\begin{proposition} \label{prop:bot-top-for-D_P}
	Let $\{v_k\}$ be an orthogonal basis of $\R^n$, and let $P\subset \R^n$ be a full-dimensional polytope.
	Then, we have that
	\[
	(D_P,\{v_k\}) \text{ has the bot-top property (resp. } \{v_k\} \text{ is }D_P\text{-admissible)}
	\]
	\[
	\iff (P\cap \rho_z P,\{v_k\}) \text{ has the bot-top property (resp. } \{v_k\} \text{ is }P\cap \rho_z P\text{-admissible)} \ \forall z \in P \ .
	\]
	Moreover, in both cases, we have that $\{v_k\}$ is $P$-admissible.
\end{proposition}
Note that we make a slight abuse of notation on the left-hand side, denoting by $D_P$ the inverse image $\iota_0^{-1}(D_P)$ of the polytope of diagonals under the isomorphism~(\ref{eq:D_P-iso}).
We first prove the following general lemma.

\begin{lemma} \label{lemma:iterated-orientation}
	Let $P\xra{\pi} Q$ be an affine projection of polytopes, and suppose that $v_1$ orients the fiber polytope $\Sigma_\pi (P,Q)$.
	Then, a vector $v_2$ perpendicular to $v_1$ orients every intersection of an hyperplane orthogonal to $v_1$ with $\Sigma_\pi (P,Q)$ if and only if this property holds for every fiber $\pi^{-1}(q), q \in Q$.
\end{lemma}

\begin{proof}
	As mentioned in the proof of \cref{lemma:orients-the-fibers}, the normal fan of $\Sigma_\pi(P,Q)$ is the common refinement of the normal fans of the fibers $\pi^{-1}(q), q \in Q$.
	The normal fan of a the intersection of an hyperplane $H$ orthogonal to $v_1$ with $\Sigma_\pi(P,Q)$ is the common refinement of the normal fans of some intersections of hyperplanes orthogonal to $v_1$ with the fibers $\pi^{-1}(q), q \in Q$.
	Indeed, as shown in \cite[Lemma 3.1]{BilleraSturmfels94}, refinement of normal fans is preserved by projection.
	We conclude as in \cref{lemma:orients-the-fibers}.
\end{proof}

\begin{proof}[Proof of \cref{prop:bot-top-for-D_P}]
	We prove the first part of the statement.
	To ease the notation, let us denote by $D$ the polytope of diagonals $D_P$.
	Using the isormophism (\ref{eq:D_P-iso}), it suffices to prove the statement for $(D,\{v_k,-v_k\})$ and $\pi^{-1}(z), z \in P$.
	It suffices to consider the case where $\{v_k,-v_k\}$ is $D$-admissible, the analogous statement with the bot-top then follows directly.
	By \cref{lemma:orients-the-fibers}, we have that $v_1$ orients $D$ if and only if it orients every fiber $\pi^{-1}(z)$.
	By the same lemma, we have that $v_2$ orients $D_1$ if and only if it orients every intersection of $D$ with an hyperplane orthogonal to $v_1$.
	According to \cref{lemma:iterated-orientation}, this is equivalent to $v_2$ orienting the intersections of each fiber $\pi^{-1}(z)$ with an hyperplane orthogonal to $v_1$.
	Using again \cref{lemma:orients-the-fibers}, this is equivalent to $v_2$ orienting $\pi^{-1}(z)_1$, for each $z \in P$.
	Continuing in the same fashion for the higher $v_k$, we get the above equivalence.

	For the second part of the statement, observe first that if $\{v_k,-v_k\}$ is $D$-admissible, then $v_k$ is $P$-admissible since the linear span $\LinSpan(F_k)$ of a $k$-face $F_k$ of $P$ is equal to the linear span of some $k$-face $F_z$ of $P\cap \rho_z P$, for some $z \in P$ (for example, $F_k \cap \rho_z F_k$ for $z=(\bm_{v_1}(F_k)+\tp_{v_1}(F_k))/2$).
	So, if all the faces of $P\cap \rho_z P$ are sent injectively to $\R^k$, then so are the faces of $P$.
	Now, suppose that $(D,\{v_k,-v_k\})$ possess the bot-top property.
	Let $F_1$ be a 1-face of $P$, with vertices $x$ and $y$.
	Setting $z=(x+y)/2$, we have that $F_1 \cap \rho_z F_1 = F_1$.
	This intersection possess the bot-top property with respect to $v_1$, so we have that $v_1$ orients $F_1$.
	Let $F_k$ by any $k$-face of $P$. \Guillaume{TBC}
\end{proof}

\Guillaume{Observation: Steenrod diagonal of P implies Steenrod diagonal on every face of P}

\Guillaume{KV choice for simplices}

\Guillaume{moduli space of Steenrod diagonals}

\Guillaume{unicity for simplices}


