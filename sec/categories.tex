% !TEX root = ../fib_poly.tex

\section{Higher categories}

In this section we prove a theorem of Kapranov--Voevodsky \cite{kapranov1991polycategory} stating that a free $n$-category structure can be constructed on any framed polytope whose associated flag is admissible.

\subsection{Globular sets, pasting diagrams and $\omega$-categories}

TBW

\subsection{Cubes and globes}

Define ... $\gcube^\infty$, $\globe^\infty$ and the map between them.

\subsection{Globularization} \label{ss:globularization}

For any $n$-dimensional polytope $P$ with a Morse frame we construct a surjective cellular map
\[
\globe^n \to P
\]
with the following equivariance property.

%and whose composition $\gcube^\infty \to \globe^\infty \to P$ is piecewise linear.

In the iterated monotone path sequence associated to $\{v_k\}$, each $P_k$ is defined via the linear functional $\angles{-,v_k}$.
We consider the iterated monotone path sequence associated to the linear functionals
\begin{equation} \label{eq:new-projection}
	\frac{\angles{-,v_k}-\angles{\bm_{v_k}(P_{k-1}),v_k}}{\angles{\tp_{v_k}(P_{k-1})-\bm_{v_k}(P_{k-1}),v_k}}
\end{equation}
where each $P_{k-1}$ is sent to the interval $\gcube$, the vertex $\bm_{v_k}(P_{k-1})$ is sent to $0$ and $\tp_{v_k}(P_{k-1})$ is sent to $1$.
We construct inductively a family of continuous piecewise linear cellular maps
\[
\Gamma_k, \Gamma_k^{\op} \colon \gcube^k \to P \ , 0 \leq k \leq n-1
\]
such that the two maps $\Gamma_k$ and $\Gamma_k^{\op}$ are both homotopies between $\Gamma_{k-1}$ and $\Gamma_{k-1}^{\op}$.
We can also extend the construction to a $\Gamma_n \colon \gcube^n \to P$ which will be an homotopy between $\Gamma_{n-1}$ and $\Gamma_{n-1}^{\op}$.

First, we define $\Gamma_0 (*) = \bm_{v_1}(P)$ and $\Gamma_0^{\op} (*) = \tp_{v_1}(P)$.
Then, the vertices $\bm_{v_2}(P_1)$ and $\tp_{v_2}(P_1)$ define uniquely sections $\Gamma_1, \Gamma_1^\op \colon \gcube \to P$ such that $\Gamma_1(0)=\Gamma_1^\op(0)=\bm_{v_1}(P)$ and $\Gamma_1(1)=\Gamma_1^\op(1)=\tp_{v_1}(P)$.
These sections are coherent monotone paths on $P$, defined explicitly by
\begin{align*}
	\Gamma_1(x) & = \bm_{v_2}(\pi_1^{-1}(x)) \\
	\Gamma_1^\op(x) & = \tp_{v_2}(\pi_1^{-1}(x))
\end{align*}
for $x \in \gcube$, where $\pi_1 \colon P \to \gcube$ denotes the projection with respect to the linear functional (\ref{eq:new-projection}) in the case $k=2$.
Next, the vertices $\bm_{v_3}(P_2)$ and $\tp_{v_3}(P_2)$ define monotone paths on $P_1$ between $\bm_{v_2}(P_1)$ and $\tp_{v_2}(P_1)$.
We consider the associated (tight coherent) subdivisions of $\gcube$, induced by $\pi_2 \colon P_1 \to \gcube$.
We denote the vertices of these subdivisions by
\begin{align*}
	x_1 & = \pi_2(\bm_{v_2}(P_1)), x_2, \ldots, x_{p-1}, x_p=\pi_2(\tp_{v_2}(P_1)) \\
	y_1 & = \pi_2(\bm_{v_2}(P_1)), y_2, \ldots, y_{q-1}, y_q=\pi_2(\tp_{v_2}(P_1)) \ .
\end{align*}
All these vertices are projections of vertices of $P_1$, which in turn have associated tight coherent sections $\Gamma_{x_i} \colon \gcube \to P$.
We define a map
\[
\Gamma_2 \colon \gcube \times \gcube \to P \ ,
\]
where the first copy of $\gcube$ receives the subdivision by the $x_i$, $1\leq i \leq p$, via the formula
\[
\Gamma_2(x_i, z) \defeq \Gamma_{x_i}(z) \ ,
\]
and on each of the cubes $[x_i,x_{i+1}] \times \gcube$ by the formula
\[
\Gamma_2(t,z) \defeq \left( \frac{x_{i+1}-t}{x_{i+1}-x_i} \right) \Gamma_{x_i}(z) + \left( \frac{t-x_i}{x_{i+1}-x_i} \right) \Gamma_{x_{i+1}}(z) \ .
\]
We proceed in a similar fashion to define $\Gamma_2^\op$ with the $y_j$.
Continuing in the same fashion, we define inductively the next maps
\[
\Gamma_k, \Gamma_k^\op \colon \gcube \times \gcube^{k-1} \to P \ , 3 \leq k \leq n-1 \ .
\]
For the last step, we observe that $P_{n-1}$ is a segment, so we can simply define
\begin{eqnarray*}
	\Gamma_n \colon \gcube \times \gcube^{n-1} & \to & P \\
	(t,z) & \mapsto & (1-t)\Gamma_{n-1}(z) + t \Gamma_{n-1}^\op (z) \ .
\end{eqnarray*}
These maps that we obtain are by construction continuous, cellular, equivariant under the action of $\Sym_2$ permuting $\Gamma_k$ and $\Gamma_k^\op$ (or sending $v_k$ to $-v_k$), and even piecewise linear.
Finally, we obtain a map from $\globe^\infty$ by defining all the higher maps trivially mapping to the point $P_n$ and taking the colimit.

\subsection{Kapranov--Voevodsky pasting theorem}

TBW