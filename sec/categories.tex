% !TEX root = ../fib_poly.tex

\section{Higher categories}

In this section we prove a theorem of Kapranov--Voevodsky \cite[Theorem 2.3]{kapranov1991polycategory} stating that a pasting scheme structure can be constructed on any locally Morse framed polytope.

\subsection{$n$-categories}

We will consider the recursive definition of $n$-\textit{categories}, that is, as categories enriched in $(n-1)$-categories.
We will take the one-sorted viewpoint identifying an $n$-category $\cC$ with a set $\Mor\cC$ together with, for $i = 0,\dots n-1$, \textit{source} and \textit{target functions}
\[
s_i, t_i \colon \Mor\cC \to \Mor\cC
\]
and \textit{partial compositions}
\[
\circ_i \colon \Mor\cC \times \Mor\cC \to \Mor\cC
\]
with $a \circ_i b$ defined when $s_i a = t_i b$.
For a complete list of relations these satisfy the reader can consult \cite[Definition 2.1]{steiner2004omega}.

\subsection{Pasting schemes}

We quote from Kapranov--Voevodsky. \cite[p.~12]{kapranov1991polycategory}
\begin{displayquote}
	Intuitively a pasting scheme is an ``algebraic expression with indeterminate elements'' which can be evaluated in an arbitrary $n$-category as soon as we have associated, in a compatible way, to the indeterminates in the expression concrete polymorphisms.
	For example,
	\begin{center}
		\input{aux/pasting}
	\end{center}
\end{displayquote}
We will work with an elaboration of their definition of pasting scheme due to Steiner \cite{steiner2004omega}; what he calls an ``augmented directed complex with a strongly loop free basis.''

A \textit{pasting scheme} is a chain complex $(\Z[B], \bd)$ of $\Z$-modules with a partially ordered basis $(B,<)$ satisfying that ...

%\subsection{Pasting diagrams and $\omega$-categories}
%
%Every (small) category has an underlying directed graph.
%The right adjoint to forgetting its composition structure is the free category construction.
%We will consider the higher version of this construction in the context of (small strict) $\omega$-categories.
%Each such $\omega$-category has an underlying (reflexive) globular set $X_\bullet$, that is, a collection of sets indexed by the non-negative integers together with maps $s_n, t_n \colon X_{n+1} \to X_n$ satisfying
%\[
%s_n \circ s_{n-1} = s_n \circ t_{n-1}.
%\]
%Notice that this notion recovers that of a directed graph if $X_n = \emptyset$ for $\geq 2$.
%
%
%Intuitively a pasting scheme is an ``algebraic expression
%with indeterminate elements'' which can be evaluated in an
%arbitrary n-category as soon as we have associated, in a com-
%patible way, to the indeterminates in the expression concrete
%polymorphisms.

\subsection{Globularization} \label{ss:globularization}

For any $n$-dimensional polytope $P$ with a Morse frame we construct surjective, cellular, piecewise linear maps
\[
\gcube^n \to \globe^n \to P
\]
which are moreover equivariant with respect to the action of $(\Sym_2)^{\times n}$.

In the iterated monotone path sequence associated to $\{v_k\}$, each $P_k$ is defined via the linear functional $\angles{-,v_k}$.
We consider the iterated monotone path sequence associated to the linear functionals
\begin{equation} \label{eq:new-projection}
	\frac{\angles{-,v_k}-\angles{\bm_{v_k}(P_{k-1}),v_k}}{\angles{\tp_{v_k}(P_{k-1})-\bm_{v_k}(P_{k-1}),v_k}}
\end{equation}
where each $P_{k-1}$ is sent to the interval $\gcube$, the vertex $\bm_{v_k}(P_{k-1})$ is sent to $0$ and $\tp_{v_k}(P_{k-1})$ is sent to $1$.
We construct inductively a family of continuous piecewise linear cellular maps
\[
\Gamma_k, \Gamma_k^{\op} \colon \gcube^k \to P \ , 0 \leq k \leq n-1
\]
such that the two maps $\Gamma_k$ and $\Gamma_k^{\op}$ are both homotopies between $\Gamma_{k-1}$ and $\Gamma_{k-1}^{\op}$.
We can also extend the construction to a $\Gamma_n \colon \gcube^n \to P$ which will be an homotopy between $\Gamma_{n-1}$ and $\Gamma_{n-1}^{\op}$.

First, we define $\Gamma_0 (*) = \bm_{v_1}(P)$ and $\Gamma_0^{\op} (*) = \tp_{v_1}(P)$.
Then, the vertices $\bm_{v_2}(P_1)$ and $\tp_{v_2}(P_1)$ define uniquely sections $\Gamma_1, \Gamma_1^\op \colon \gcube \to P$ such that $\Gamma_1(0)=\Gamma_1^\op(0)=\bm_{v_1}(P)$ and $\Gamma_1(1)=\Gamma_1^\op(1)=\tp_{v_1}(P)$.
These sections are coherent monotone paths on $P$, defined explicitly by
\begin{align*}
	\Gamma_1(x) & = \bm_{v_2}(\pi_1^{-1}(x)) \\
	\Gamma_1^\op(x) & = \tp_{v_2}(\pi_1^{-1}(x))
\end{align*}
for $x \in \gcube$, where $\pi_1 \colon P \to \gcube$ denotes the projection with respect to the linear functional (\ref{eq:new-projection}) in the case $k=2$.
Next, the vertices $\bm_{v_3}(P_2)$ and $\tp_{v_3}(P_2)$ define monotone paths on $P_1$ between $\bm_{v_2}(P_1)$ and $\tp_{v_2}(P_1)$.
We consider the associated (tight coherent) subdivisions of $\gcube$, induced by $\pi_2 \colon P_1 \to \gcube$.
We denote the vertices of these subdivisions by
\begin{align*}
	x_1 & = \pi_2(\bm_{v_2}(P_1)), x_2, \ldots, x_{p-1}, x_p=\pi_2(\tp_{v_2}(P_1)) \\
	y_1 & = \pi_2(\bm_{v_2}(P_1)), y_2, \ldots, y_{q-1}, y_q=\pi_2(\tp_{v_2}(P_1)) \ .
\end{align*}
All these vertices are projections of vertices of $P_1$, which in turn have associated tight coherent sections $\Gamma_{x_i} \colon \gcube \to P$.
We define a map
\[
\Gamma_2 \colon \gcube \times \gcube \to P \ ,
\]
where the second copy of $\gcube$ receives the subdivision by the $x_i$, $1\leq i \leq p$, via the formula
\[
\Gamma_2(z, x_i) \defeq \Gamma_{x_i}(z) \ ,
\]
and on each of the cubes $\gcube \times [x_i,x_{i+1}]$ by the formula
\[
\Gamma_2(z,t) \defeq \left( \frac{x_{i+1}-t}{x_{i+1}-x_i} \right) \Gamma_{x_i}(z) + \left( \frac{t-x_i}{x_{i+1}-x_i} \right) \Gamma_{x_{i+1}}(z) \ .
\]
We proceed in a similar fashion to define $\Gamma_2^\op$ with the $y_j$.
Continuing in the same fashion, we define inductively the next maps
\[
\Gamma_k, \Gamma_k^\op \colon \gcube^{k-1} \times \gcube \to P \ , 3 \leq k \leq n-1 \ .
\]
For the last step, we observe that $P_{n-1}$ is a segment, so we can simply define
\begin{eqnarray*}
	\Gamma_n \colon  \gcube^{n-1} \times \gcube  & \to & P \\
	(z,t) & \mapsto & (1-t)\Gamma_{n-1}(z) + t \Gamma_{n-1}^\op (z) \ .
\end{eqnarray*}
These maps that we obtain are by construction continuous, cellular, and even piecewise linear.
We will use them to define two surjective cellular maps
\[
	\gcube^n \xra{\Psi} \globe^n \xra{\Upsilon} P \ .
\]
The fist map $\Psi=(\Psi_1,\ldots,\Psi_n)$ is defined component-wise by the formulas
\begin{align*}
	\Psi_1(t_1,\ldots,t_n) & \defeq \cos(t_1 \pi) \\
	\Psi_2(t_1,\ldots,t_n) & \defeq \sin(t_1 \pi)\cos(t_2\pi) \\
	\Psi_3(t_1,\ldots,t_n) & \defeq \sin(t_1 \pi)\sin(t_2\pi)\cos(t_3\pi) \\
	 & \vdots \\
	\Psi_{n-1}(t_1,\ldots,t_n) & \defeq \sin(t_1 \pi) \cdots \sin(t_{n-2} \pi) \cos(t_{n-1} \pi) \\
	\Psi_n(t_1,\ldots,t_n) & \defeq (1-2t_n) \sin(t_1 \pi) \cdots \sin(t_{n-1} \pi) \ ,
\end{align*}
while the second map $\Upsilon$ is defined by
\[
\Upsilon(\Psi(t_1,\ldots,t_n)) \defeq \Gamma_n(t_1,\ldots, t_n)\ .
\]
By construction, these maps are again continuous, cellular and piecewise linear.
There is moreover an action of $(\Sym_2)^{\times n}$ on each of the three spaces
\begin{itemize}
	\item On the $n$-cube $\gcube^n$ by sending a coordinate $t_i$ to $1-t_i$, its reflection with respect to the value $1/2$.
	\item On the $n$-globe $\globe^n$ by sending a coordinate $\Psi_i$ to its opposite $-\Psi_i$.
	\item On the framed polytope $P$ by sending a vector $v_i$ to its opposite $-v_i$, or equivalently sending the map $\Gamma_{i-1}$ to its opposite $\Gamma_{i-1}^\op$.
\end{itemize}
One can check that the two maps $\Psi$ and $\Upsilon$ are equivariant with respect to this action.

\subsection{Kapranov--Voevodsky pasting scheme theorem}

TBW