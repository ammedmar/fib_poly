% !TEX root = ../fib_poly.tex

\section{Higher categories}

Observe that if $(P,\{v_k\})$ is a framed polytope, then for every face $F$ of $P$, the pair $(F,\{v_k\})$ is a framed polytope.

\begin{definition}
	A framed polytope $(P,\{v_k\})$ possess the \emph{local bot-top property} if every face $(F,\{v_k\})$ of $P$ possess the bot-top property.
\end{definition}

\begin{definition}
	A frame $\{v_k\}$ is \emph{$P$-admissible} if every $k$-face of $P$ is sent injectively to $\R^k$ by the projection projection $\pi_k$ in the associated flag.
\end{definition}

\begin{lemma} \label{l:P-admissible}
	A frame $\{v_k\}$ is $P$-admissible if and only if for any $k$-dimensional face $F_k$ of $P$, the vector $v_k$ is not perpendicular to the line $\LinSpan(F_k) \cap \ker \pi_{k-1}$.
\end{lemma}

\begin{proof}
	Requiring $\pi_1$ to be injective on the 1-faces of $P$ is exactly asking that for every 1-face $F_1$, we have $\LinSpan(F_1)\cap \ker \pi_1=0$, which is the same as requiring that $v_1$ is not perpendicular to $\LinSpan(F_1)=\LinSpan(F_1)\cap \ker \pi_0$.
	Note that on both sides, it is not possible that the condition for $k=2$ is satisfied while the condition for $k=1$ does not hold.
	So, supposing that this last condition holds, we have indeed that for all 2-face $F_2$ of $P$, $\LinSpan(F_2)\cap \ker \pi_1$ is a line (the dimension of $\LinSpan(F_2)\cap \ker \pi_1$ is at least 1, and if it was 2 it would mean that some 1-faces of $F_2$ are perpendicular to $v_1$).
	Requiring that $\pi_2$ is injective on 2-faces of $P$ is exactly asking that for every 2-face $F_2$ of $P$, we have $\LinSpan(F_2)\cap \ker \pi_2=0$, which is the same as requiring that $v_2$ is not perpendicular to $\LinSpan(F_2)\cap \ker \pi_1$.
	Continuing in the same fashion, we get the equivalence between the injectivity condition on both sides.
\end{proof}

\begin{lemma} \label{l:bot-top-admissible}
	A framed polytope $(P,\{v_k\})$ possess the local bot-top property if and only if the frame $\{v_k\}$ is $P$-admissible.
\end{lemma}

\begin{proof}
	We will use the characterization of $P$-admissibility obtained in \cref{l:P-admissible}.
	For a $1$-face $F_1$ of $P$, we have by definition that the bot-top condition with respect to $v_1$ is equivalent to the admissibility condition.
	For a $2$-face $F_2$ of $P$, the bot-top condition says that $v_2$ should not be perpendicular to any intersection of $F_2$  with an hyperplane perpendicular to $v_1$, which is exactly requiring that $v_2$ is not perpendicular to $\LinSpan(F_2)\cap \ker \pi_1$.
	The proof for the higher dimensional faces of $P$ is similar.
\end{proof}

\begin{definition}
	A vector $v$ \emph{orients} a polytope $P$ if this vector is not orthogonal to any edge of $P$.
\end{definition}

\begin{lemma} \label{l:orients-the-fibers}
	Let $P \xra{\pi} Q$ be a projection of polytopes.
	A vector $v$ orients the fiber polytope $\Sigma_\pi(P, Q)$ if and only if it orients every fiber $\pi^{-1}(q)$.
\end{lemma}

\begin{proof}
	The normal fan of $\Sigma_\pi(P,Q)$ is the common refinement of the normal fans of the fibers $\pi^{-1}(q), q \in Q$ \cite[Proposition 2.2]{BilleraSturmfels94}.
	It follows that the set of directions of the edges of $\Sigma_\pi(P,Q)$ is the union of the sets of directions of the edges of $\pi^{-1}(q), q \in Q$.
	Indeed, the edge directions of $\Sigma_\pi(P,Q)$ are normal directions to the walls of $\mathcal{N}_{\Sigma_\pi(P,Q)}$, which are precisely covered by the union of the walls of $\mathcal{N}_{\pi^{-1}(q)}, q \in Q$.
\end{proof}

\begin{lemma} \label{l:KV-BS}
	A frame $\{v_k\}$ is $P$-admissible if and only if, in the associated iterated monotone path polytope $P_\bullet$, for all $1 \leq k \leq n$ the polytope $P_{k-1}$ is oriented by $v_k$.
\end{lemma}

\begin{proof}
	From the preceding lemma, we get that the lines generated by the edges of $P_{k-1}$ are exactly the lines $\LinSpan(F_k)\cap \ker \pi_{k-1}$, for $F_k$ a $k$-face of $P$.
	By \cref{l:P-admissible}, $P$-admissibility is equivalent to requiring that $v_k$ is not perpendicular to these lines, which is then (by definition) requiring that $v_k$ orients~$P_{k-1}$.
\end{proof}

We observe that if $v$ orients $P$, then every face $F$ of $P$ has a unique vertex $\bm_v(F)$ (resp. $\tp_v(F)$) which minimizes (resp. maximizes) the scalar product $\angles{-,v}$.

\begin{theorem}	\label{t:bot-top-for-fibers}
	Let $P \xra{\pi} Q$ be a projection of polytopes, and let $\{v_k\}$ be an orthonormal basis of $\ker \pi$.
	\anibal{Kernel? It is not assumed that the barycenter of $Q$ is the origin.}
	Then, we have that
	\[
	(\Sigma_\pi(P,Q),\{v_k\}) \text{ has the bot-top property (resp. } \{v_k\} \text{ is }\Sigma_\pi(P,Q)\text{-admissible)}
	\]
	\[
	\iff (\pi^{-1}(q),\{v_k\}) \text{ has the bot-top property (resp. } \{v_k\} \text{ is }\pi^{-1}(q)\text{-admissible)} \ \forall q \in Q \ .
	\]
\end{theorem}

We first prove the following general lemma.
\anibal{Consider not having this lemma as a stand alone since it is not used anywhere else, and instead absorve it and its proof into the proof of \cref{t:bot-top-for-fibers}}.

\begin{lemma} \label{l:iterated-orientation}
	Let $P\xra{\pi} Q$ be an affine projection of polytopes, let $v_1, v_2$ be two orthogonal unit vectors in $\ker \pi$ and suppose that $v_1$ orients $\Sigma_\pi (P,Q)$.
	Then, the vector $v_2$ orients every intersection of $\Sigma_\pi (P,Q)$ with an hyperplane orthogonal to $v_1$ if and only if $v_2$ orients every intersection of $\pi^{-1}(q)$ with an hyperplane orthogonal to $v_1$, for all $q \in Q$.
\end{lemma}

\begin{proof}
	As mentioned in the proof of \cref{l:orients-the-fibers}, the normal fan of $\Sigma_\pi(P,Q)$ is the common refinement of the normal fans of the fibers $\pi^{-1}(q), q \in Q$.
	The normal fan of the intersection of $\Sigma_\pi(P,Q)$ with an hyperplane orthogonal to $v_1$ is the common refinement of the normal fans of the intersections of the fibers $\pi^{-1}(q), q \in Q$ with some hyperplanes orthogonal to $v_1$.
	Indeed, as shown in \cite[Lemma 3.1]{BilleraSturmfels94}, refinement of normal fans is preserved by projection.
	The result follows as in \cref{l:orients-the-fibers}.
\end{proof}

\begin{proof}[Proof of \cref{t:bot-top-for-fibers}]
	To ease the notation, let us denote by $\Sigma$ the fiber polytope $\Sigma(P,Q)$.
	It suffices to consider the case where $\{v_k\}$ is $\Sigma$-admissible, the analogous statement with the bot-top then follows directly.
	By \cref{l:orients-the-fibers}, we have that $v_1$ orients $\Sigma$ if and only if it orients every fiber $\pi^{-1}(q), q \in Q$.
	By the same lemma, we have that $v_2$ orients $\Sigma_1$ if and only if it orients every intersection of $\Sigma$ with an hyperplane orthogonal to $v_1$.
	According to \cref{l:iterated-orientation}, this is equivalent to $v_2$ orienting the intersections of each fiber $\pi^{-1}(q)$ with any hyperplane orthogonal to $v_1$.
	Using again \cref{l:orients-the-fibers}, this is equivalent to $v_2$ orienting $\pi^{-1}(q)_1$, for each $q \in Q$.
	Continuing in the same fashion for the higher $v_k$, we get the above equivalence.
\end{proof}
