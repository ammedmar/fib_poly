% !TEX root = ../fib_poly.tex

\section{Iterated monotone path polytopes}

\Guillaume{We work in the linear setting; needs to translate into affine}

\subsection{Flags and frames}

A \emph{flag} is a diagram of full rank linear maps
\[
\R^n \xra{\pi^n_{n-1}} \R^{n-1} \xra{\pi^{n-1}_{n-2}} \dotsb \xra{\pi^3_{2}} \R^2 \xra{\pi^2_1} \R^1 \xra{\pi^1_0} \R^0.
\]
It is completely determined by the compositions
$\pi_k \colon \R^n \to \R^k$.
Given an ordered orthonormal basis $B = \{v_i\}$ of $\R^n$, which we will refer to as a \textit{frame}, its \emph{associated flag} $\pi_\bullet = \{\pi_k\}$ is defined by forgetting the last coordinates in this basis.
Two frames give rise to the same associated flag if and only if they differ only on signs, and, given a flag, the Gram--Schmidt process can be used to construct a frame having it as its associated one.

%\begin{lemma}
%	The data of a full-rank flag is equivalent, up to sign, to the data of an orthonormal basis $\{v_1,\ldots,v_n\}$ of $\R^n$ such that $\ker \pi_k=\LinSpan(v_n, \ldots, v_{k+1})$ for all $k$.
%\end{lemma}
%
%\begin{proof}
%	Since $\pi_{n-1}$ has full rank, its kernel is 1-dimensional.
%	We can choose a unit vector $v_n$ such that $\ker \pi_{n-1} = \LinSpan(v_n)$.
%	The kernel of $\pi_{n-2}$ is 2-dimensional, and contains $\ker \pi_{n-1}$, so we can choose a unit vector $v_{n-1}$, perpendicular to $v_n$, such that $\ker \pi_{n-2}=\LinSpan(v_n,v_{n-1})$.
%	Continuing in the same fashion, we obtain an ordered basis $\{v_1,\ldots, v_n\}$ of $\R^n$ such that $\ker \pi_{k}=\LinSpan(v_n,\ldots,v_{k+1})$.
%\end{proof}

\subsection{Fiber polytopes}

A polytope $P \subset \mathbb{R}^n$ is the convex hull of a finite set of points.
Equivalently, it is defined as the bounded intersection of a finite number of half-spaces.

Let $\pi \colon P \to Q$ be a projection of polytopes, that is, an affine map $\pi \colon \R^p \to \R^q$ such that $\pi(P)=Q$ for two polytopes $P\subset \R^p$ and $Q \subset \R^q$.
One can associate to it the \emph{fiber polytope} \[\Sigma(P,Q) \defeq \left\{ \frac{1}{vol(Q)}\int_Q \gamma(x)dx \ \colon \ \gamma \text{ is a section of }\pi \right\} \ , \] whose face lattice is isomorphic to the lattice of $\pi$-coherent subdivisions of $Q$ \cite{BilleraSturmfels92}.
See also \cite[Chapter 9]{Ziegler95} for a detailed account.
The vertices of $\Sigma(P,Q)$ correspond to tight coherent subdivisions, that is, subdivisions $\pi(\mathcal{F})$ where for each face $F \in \mathcal{F}\subset\mathcal{L}(P)$ we have $\dim(\pi(F))=\dim(F)$.
Any linear function $c\in (\R^p)^{*}$ induces a subdivision $\pi(\mathcal{F}^c)$.
When this subdivision is tight, there is a unique section $\gamma^c$ of $\pi$ which maximizes $c$ in each fiber.

Moreover, the fiber polytope $\Sigma(P,Q)$ lives in the preimage of the barycenter $r$ of $Q$, that is, $\Sigma(P,Q)\subset\pi^{-1}(r)\cap P$.

\subsection{Framed polytopes and the iterated monotone path polytopes}

A \textit{framed polytope} is the convex hull of a finite set of point in $\R^n$ together with an orthonormal basis of $\R^n$.
Its \textit{associated flag} is the the one associated to its basis.

\begin{definition}
	Let $P$ be a framed polytope with associated flag $\pi_\bullet$.
	Its \textit{iterated monotone path polytope} $P_\bullet = \{P_k\}_{k \geq 0}$ is the collection of framed polytopes defined recursively by the formulas
	\begin{align*}
		P_0 & \defeq P \\
		P_1 & \defeq \Sigma_{\pi_1}(P_0,\pi_1(P_0)) \\
		P_k & \defeq \Sigma_{\pi_k}(P_{k-1},\Sigma_{\pi^{k}_{k-1}}(P_{k-2})), \ 2 \leq k \leq n \ ,
	\end{align*}
	where $\Sigma_{\pi^{k}_{k-1}}(P_{k-2})\defeq \Sigma_{\pi^{k}_{k-1}}(\pi_k(P_{k-2}),\pi_{k-1}(P_{k-2}))$.
\end{definition}

Note that by definition of the fiber polytope, we have $P_k \subset \ker \pi_k$ and $\dim P_k = \dim P - \min\{ \dim P,k\}$ for all $0\leq k \leq n$.
In the case where $P$ is full dimensional, we have $\dim P_k = n-k$, and all the $P_k, 1 \leq k \leq n$ are made out of a projection to a 1-dimensional polytope, so they are \emph{monotone path polytopes} \cite[Theorem 5.3]{BilleraSturmfels92}.
Note that flags of polytopes and their associated iterated fiber polytopes where studied in \cite{BilleraSturmfels94}.
Note also that in order for this definition to make sense, one needs to use crucially the functoriality of the fiber polytope construction \cite[Lemma 2.3]{BilleraSturmfels92}.
\anibal{Why?}

\subsection{Zonotopes} \label{ss:zonotopes}

IF $P$ is a zonotope, then all the $P_k$ are also zonotopes.
Indeed, in this case $P$ is the projection of a cube $C \xra{\pi} P$.
\cite[Theorem 4.1]{BilleraSturmfels92} says that the fiber polytope $\Sigma(C,P)$ of this projection is itself a zonotope, the \emph{fiber zonotope} of $P$.
We consider the flag $C \xra{\pi} P \xra{\pi_1} I$, where $I$ is the $1$-dimensional image of $P$ under the projection $\pi_1$.
Using the functoriality property \cite[Lemma 2.3]{BilleraSturmfels92}, we have that $\Sigma(P,I)=\pi(\Sigma(C,I))$ is the projection of a zonotope, thus itself a zonotope.
Repeating this argument, we get that all the $P_k$ are zonotopes.
\anibal{What is a zonotope? Why are they interesting?}

\subsection{Centrally symmetric polytopes} \label{ss:centrally-symmetric}

If $P$ is centrally symmetric, then all the $P_k$ are also centrally symmetric.
Indeed, in this case $P$ is the projection of a cross-polytope $T \xra{\pi} P$.
\cite[Theorem 5.1]{BilleraSturmfels92} says that the fiber polytope $\Sigma(T,P)$ of this projection is itself centrally symmetric.
A similar argument to \cref{ss:zonotopes} yields the result.
\anibal{Is this result used in the Steenrod diagonal section? If so, consider a definition of centrally symmetric and a lemma proving this statement.}

\subsection{Bot-top property}

\anibal{I am still thinking of a different name for this}

We say that $(P,v)$ possess the \emph{bot-top property} if it has a unique vertex $\bm_v(P)$ which minimizes the scalar product $\angles{-,v}$ and a unique vertex $\tp_v(P)$ which maximizes it.
We say that $(P,\{v_k\})$ possess the \emph{bot-top property} if $(P_{k-1}, v_k)$ possess it, for every $1\leq k \leq n$.

The bot-top property for $(P,v)$ was named \emph{quasi-orientation} in \cite[Definition 1.11]{GLA21}.

\subsection{Standard globe}

\anibal{In the Steenrod section you have a piecewise linear map from the cube, which is appealing because of the rigidity of such maps. Maybe the statement is that it factors through the/a projection from the cube to the standard globe?}

\begin{theorem}
	\label{thm:map-from-the-globe}
	If a framed polytope $(P,\{v_k\})$ has the bot-top property, then there is an equivariant cellular map $\mathbb{G}^\infty \to P$.
\end{theorem}

\begin{proof}
	We suppose that $P$ has dimension $n\geq 0$.
	In the iterated monotone path polytope associated to $\{v_k\}$, each $P_k$ is defined via the linear functional $\angles{-,v_k}$.
	We consider the iterated monotone path polytope associated to the linear functionals
	\begin{equation} \label{eq:new-projection}
		\frac{\angles{-,v_k}-\angles{\bm_{v_k}(P_{k-1}),v_k}}{\angles{\tp_{v_k}(P_{k-1})-\bm_{v_k}(P_{k-1}),v_k}}
	\end{equation}
	where each $P_{k-1}$ is sent to the interval $I$, the vertex $\bm_{v_k}(P_{k-1})$ is sent to $0$ and $\tp_{v_k}(P_{k-1})$ is sent to $1$.
	We construct inductively a family of continuous piece-wise linear cellular maps
	\[
	\triangle_k, \triangle_k^{\op}: I^k \to P \ , 0 \leq k \leq n-1
	\]
	such that the two maps $\triangle_k$ and $\triangle_k^{\op}$ are both homotopies between $\triangle_{k-1}$ and $\triangle_{k-1}^{\op}$.
	We can also extend the construction to a $\triangle_n : I^n \to P$ which will be an homotopy between $\triangle_{n-1}$ and $\triangle_{n-1}^{\op}$.

	First, we define $\triangle_0 (*) \defeq \bm_{v_1}(P)$ and $\triangle_0^{\op} (*) \defeq \tp_{v_1}(P)$.
	Then, the vertices $\bm_{v_2}(P_1)$ and $\tp_{v_2}(P_1)$ define uniquely sections $\triangle_1, \triangle_1^\op : I \to P$ such that $\triangle_1(0)=\triangle_1^\op(0)=\bm_{v_1}(P)$ and $\triangle_1(1)=\triangle_1^\op(1)=\tp_{v_1}(P)$.
	These sections are coherent monotone paths on $P$, defined explicitly by
	\begin{align*}
		\triangle_1(x) & = \bm_{v_2}(\pi_1^{-1}(x)) \\
		\triangle_1^\op(x) & = \tp_{v_2}(\pi_1^{-1}(x))
	\end{align*}
	for $x \in I$, where $\pi_1 : P \to I$ denotes the projection with respect to the linear functional (\ref{eq:new-projection}) in the case $k=2$.
	Next, the vertices $\bm_{v_3}(P_2)$ and $\tp_{v_3}(P_2)$ define monotone paths on $P_1$ between $\bm_{v_2}(P_1)$ and $\tp_{v_2}(P_1)$.
	We consider the associated (tight coherent) subdivisions of $I$, induced by $\pi_2 : P_1 \to I$.
	We denote the vertices of these subdivisions by
	\begin{align*}
		x_1 & =\pi_2(\bm_{v_2}(P_1)), x_2, \ldots, x_{p-1}, x_p=\pi_2(\tp_{v_2}(P_1)) \\
		y_1 & = \pi_2(\bm_{v_2}(P_1)), y_2, \ldots, y_{q-1}, y_q=\pi_2(\tp_{v_2}(P_1)) \ .
	\end{align*}
	All these vertices are projections of vertices of $P_1$, which in turn have associated tight coherent sections $\triangle_{x_i}: I \to P$.
	We define a map
	\[
	\triangle_2 : I \times I \to P \ ,
	\]
	where the first copy of $I$ receives the subdivision by the $x_i$, $1\leq i \leq p$, via the formula
	\[
	\triangle_2(x_i, z) \defeq \triangle_{x_i}(z) \ ,
	\]
	and on each of the cubes $[x_i,x_{i+1}]\times I$ by the formula
	\[
	\triangle_2(t,z) \defeq \left( \frac{x_{i+1}-t}{x_{i+1}-x_i} \right) \triangle_{x_i}(z) + \left( \frac{t-x_i}{x_{i+1}-x_i} \right) \triangle_{x_{i+1}}(z) \ .
	\]
	We proceed in a similar fashion to define $\triangle_2^\op$ with the $y_j$.
	Continuing in the same fashion, we define inductively the next maps
	\[
	\triangle_k, \triangle_k^\op : I \times I^{k-1} \to P \ , 3 \leq k \leq n-1 \ .
	\]
	For the last step, we observe that $P_{n-1}$ is a segment, so we can simply define
	\begin{eqnarray*}
		\triangle_n : I \times I^{n-1} & \to & P \\
		(t,z) & \mapsto & (1-t)\triangle_{n-1}(z) + t \triangle_{n-1}^\op (z) \ .
	\end{eqnarray*}
	These maps that we obtain are by construction continuous, cellular, equivariant under the action of $\Sym_2$ permuting $\triangle_k$ and $\triangle_k^\op$ (or sending $v_k$ to $-v_k$), and even piece-wise linear.
	Finally, we obtain a map from $\mathbb{G}^\infty$ by defining all the higher maps trivially mapping to the point $P_n$ and taking the colimit.
\end{proof}
