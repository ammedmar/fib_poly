% !TEX root = ../fib_poly.tex

\section{Framed polytopes}

\Guillaume{We work in the linear setting; needs to translate into affine}

\subsection{Flags and frames}

A \emph{flag} is a diagram of full rank linear maps
\[
\R^n \xra{\pi^n_{n-1}} \R^{n-1} \xra{\pi^{n-1}_{n-2}} \dotsb \xra{\pi^3_{2}} \R^2 \xra{\pi^2_1} \R^1 \xra{\pi^1_0} \R^0.
\]
It is completely determined by the compositions
$\pi_k \colon \R^n \to \R^k$.
Given an ordered orthonormal basis $B = \{v_i\}$ of $\R^n$, which we will refer to as a \textit{frame}, its \emph{associated flag} $\pi_\bullet = \{\pi_k\}$ is defined by forgetting the last coordinates in this basis.
Two frames give rise to the same associated flag if and only if they differ only on signs, and, given a flag, the Gram--Schmidt process can be used to construct a frame having it as its associated one.

%\begin{lemma}
%	The data of a full-rank flag is equivalent, up to sign, to the data of an orthonormal basis $\{v_1,\ldots,v_n\}$ of $\R^n$ such that $\ker \pi_k=\LinSpan(v_n, \ldots, v_{k+1})$ for all $k$.
%\end{lemma}
%
%\begin{proof}
%	Since $\pi_{n-1}$ has full rank, its kernel is 1-dimensional.
%	We can choose a unit vector $v_n$ such that $\ker \pi_{n-1} = \LinSpan(v_n)$.
%	The kernel of $\pi_{n-2}$ is 2-dimensional, and contains $\ker \pi_{n-1}$, so we can choose a unit vector $v_{n-1}$, perpendicular to $v_n$, such that $\ker \pi_{n-2}=\LinSpan(v_n,v_{n-1})$.
%	Continuing in the same fashion, we obtain an ordered basis $\{v_1,\ldots, v_n\}$ of $\R^n$ such that $\ker \pi_{k}=\LinSpan(v_n,\ldots,v_{k+1})$.
%\end{proof}

\subsection{Fiber polytopes}

A polytope $P \subset \mathbb{R}^n$ is the convex hull of a finite set of points.
Equivalently, it is defined as the bounded intersection of a finite number of half-spaces.

Let $\pi \colon P \to Q$ be a projection of polytopes, that is, an affine map $\pi \colon \R^p \to \R^q$ such that $\pi(P)=Q$ for two polytopes $P\subset \R^p$ and $Q \subset \R^q$.
One can associate to it the \emph{fiber polytope} \[\Sigma(P,Q) \defeq \left\{ \frac{1}{vol(Q)}\int_Q \gamma(x)dx \ \colon \ \gamma \text{ is a section of }\pi \right\} \ , \] whose face lattice is isomorphic to the lattice of $\pi$-coherent subdivisions of $Q$ \cite{BilleraSturmfels92}.
See also \cite[Chapter 9]{Ziegler95} for a detailed account.
The vertices of $\Sigma(P,Q)$ correspond to tight coherent subdivisions, that is, subdivisions $\pi(\mathcal{F})$ where for each face $F \in \mathcal{F}\subset\mathcal{L}(P)$ we have $\dim(\pi(F))=\dim(F)$.
Any linear function $c\in (\R^p)^{*}$ induces a subdivision $\pi(\mathcal{F}^c)$.
When this subdivision is tight, there is a unique section $\gamma^c$ of $\pi$ which maximizes $c$ in each fiber.

Moreover, the fiber polytope $\Sigma(P,Q)$ lives in the preimage of the barycenter $r$ of $Q$, that is, $\Sigma(P,Q)\subset\pi^{-1}(r)\cap P$.

\subsection{Monotone path polytopes}

TBW example of fiber poly...

\subsection{Framed polytope}

A \textit{framed polytope} is the convex hull of a finite set of point in $\R^n$ together with an orthonormal basis of $\R^n$.
Observe that if $(P, \{v_k\})$ is a framed polytope, then for every face $F$ of $P$, the pair $(F, \{v_k\})$ is a framed polytope.
Its \textit{associated flag} is the the one associated to its basis.

\begin{definition}
	Let $P$ be a framed polytope with associated flag $\pi_\bullet$.
	Its \textit{iterated monotone path polytope} $P_\bullet = \{P_k\}_{k \geq 0}$ is the collection of framed polytopes defined recursively by the formulas
	\begin{align*}
		P_0 & \defeq P \\
		P_1 & \defeq \Sigma_{\pi_1}(P_0,\pi_1(P_0)) \\
		P_k & \defeq \Sigma_{\pi_k}(P_{k-1},\Sigma_{\pi^{k}_{k-1}}(P_{k-2})), \ 2 \leq k \leq n \ ,
	\end{align*}
	where $\Sigma_{\pi^{k}_{k-1}}(P_{k-2})\defeq \Sigma_{\pi^{k}_{k-1}}(\pi_k(P_{k-2}),\pi_{k-1}(P_{k-2}))$.
\end{definition}

Note that by definition of the fiber polytope, we have $P_k \subset \ker \pi_k$ and $\dim P_k = \dim P - \min\{ \dim P,k\}$ for all $0\leq k \leq n$.
In the case where $P$ is full dimensional, we have $\dim P_k = n-k$, and all the $P_k, 1 \leq k \leq n$ are made out of a projection to a 1-dimensional polytope, so they are \emph{monotone path polytopes} \cite[Theorem 5.3]{BilleraSturmfels92}. \anibal{This should be moved to the previous subsection.}
Note that flags of polytopes and their associated iterated fiber polytopes where studied in \cite{BilleraSturmfels94}.
Note also that in order for this definition to make sense, one needs to use crucially the functoriality of the fiber polytope construction \cite[Lemma 2.3]{BilleraSturmfels92}.
\anibal{Why?}

\subsection{Zonotopes} \label{ss:zonotopes}

IF $P$ is a zonotope, then all the $P_k$ are also zonotopes.
Indeed, in this case $P$ is the projection of a cube $C \xra{\pi} P$.
\cite[Theorem 4.1]{BilleraSturmfels92} says that the fiber polytope $\Sigma(C,P)$ of this projection is itself a zonotope, the \emph{fiber zonotope} of $P$.
We consider the flag $C \xra{\pi} P \xra{\pi_1} I$, where $I$ is the $1$-dimensional image of $P$ under the projection $\pi_1$.
Using the functoriality property \cite[Lemma 2.3]{BilleraSturmfels92}, we have that $\Sigma(P,I)=\pi(\Sigma(C,I))$ is the projection of a zonotope, thus itself a zonotope.
Repeating this argument, we get that all the $P_k$ are zonotopes.
\anibal{What is a zonotope? Why are they interesting?}

\subsection{Centrally symmetric polytopes} \label{ss:centrally-symmetric}

If $P$ is centrally symmetric, then all the $P_k$ are also centrally symmetric.
Indeed, in this case $P$ is the projection of a cross-polytope $T \xra{\pi} P$.
\cite[Theorem 5.1]{BilleraSturmfels92} says that the fiber polytope $\Sigma(T,P)$ of this projection is itself centrally symmetric.
A similar argument to \cref{ss:zonotopes} yields the result.
\anibal{I think this result used in the Steenrod diagonal section. Consider expanding and including a lemma env. proving this statement.}


