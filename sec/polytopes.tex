% !TEX root = ../fib_poly.tex

\section{Framed polytopes}

\Todo{We work in the linear setting; needs to translate into affine}

\subsection{Fiber polytopes}

A polytope $P \subset \mathbb{R}^n$ is the convex hull of a finite set of points.
Equivalently, it is defined as the bounded intersection of a finite number of half-spaces.

Let $\pi \colon P \to Q$ be a projection of polytopes, that is, an affine map $\pi \colon \R^p \to \R^q$ such that $\pi(P)=Q$ for two polytopes $P\subset \R^p$ and $Q \subset \R^q$.
One can associate to it the \emph{fiber polytope} \[\Sigma(P,Q) \defeq \left\{ \frac{1}{vol(Q)}\int_Q \gamma(x)dx \ \colon \ \gamma \text{ is a section of }\pi \right\} \ , \] whose face lattice is isomorphic to the lattice of $\pi$-coherent subdivisions of $Q$ \cite{BilleraSturmfels92}.
See also \cite[Chapter 9]{Ziegler95} for a detailed account.
The vertices of $\Sigma(P,Q)$ correspond to tight coherent subdivisions, that is, subdivisions $\pi(\mathcal{F})$ where for each face $F \in \mathcal{F}\subset\mathcal{L}(P)$ we have $\dim(\pi(F))=\dim(F)$.
Any linear function $c\in (\R^p)^{*}$ selects a family of faces $\mathcal{F}^c$ of $P$, the faces on which this function is maximal. 
This family induces a subdivision $\pi(\mathcal{F}^c)$ of $Q$.
When this subdivision is tight, there is a unique section $\gamma^c$ of $\pi$ which maximizes $c$ in each fiber.

Moreover, the fiber polytope $\Sigma(P,Q)$ lives in the preimage of the barycenter $r$ of $Q$, that is, $\Sigma(P,Q)\subset\pi^{-1}(r)\cap P$.

\subsection{Monotone path polytopes}

When $Q=[a,b]$ is $1$-dimensional, the fiber polytope $\Sigma(P,Q)$ is called the \emph{monotone path polytope} of $P$ \cite[Theorem 5.3]{BilleraSturmfels92}. 
It has dimension $\dim P -1$, and its vertices encode monotone paths between the extremal faces $\pi^{-1}(a)$ and $\pi^{-1}(b)$, that is sequences of edges for which a linear function $c \in (\R^p)^{*}$ is strictly increasing. 
One usually chooses a \emph{generic} function $c$, that is a function which is non-constant on every edge of $P$. 
For instance, the monotone path polytope of the $n$-simplex is the $(n-1)$-cube, and the monotone path polytope of the $(n-1)$-cube is the $(n-2)$-dimensional permutahedron (\Guillaume{Ref}). 
The monotone path polytope construction led to the resolution of the \emph{Baues conjecture} (\Guillaume{Ref: Billera-Sturfmels-Kapranov + Reiner Survey}), a problem which originated algebraic topology, in relation to the study of iterated loop spaces (...). \Guillaume{This might fit better in the introduction...}

\subsection{Adjunction}

Given a piecewise linear map $R \to \Sigma(P,Q)$, we construct a piecewise linear map $R \times Q \to P$ such that ... ? \Guillaume{Is $\Sigma(-,-)$ a bifunctor? Could $\times$ be an adjoint, maybe by averaging the maps $P \times Q \to R$? $\Sigma(P,Q)$ is obtained by averaging the fibers}

\subsection{Framed polytope}

A \textit{framed polytope} is a polytope $P \subset \R^n$ together with a frame of $\R^n$.
Observe that if $(P, \{v_k\})$ is a framed polytope, then for every face $F$ of $P$, the pair $(F, \{v_k\})$ is a framed polytope.
%We refer to the flag associated to the frame $\{v_i\}$ as that associated to $P$.
Let $P$ be a framed polytope with associated flag $\pi_\bullet$.
Its \textit{iterated monotone path sequence} $P_\bullet = \{P_k\}_{k \geq 0}$ is the collection of framed polytopes defined recursively by the formulas
\begin{align*}
	P_0 & \defeq P \\
	P_1 & \defeq \Sigma_{\pi_1}(P_0,\pi_1(P_0)) \\
	P_k & \defeq \Sigma_{\pi_k}(P_{k-1},\Sigma_{\pi^{k}_{k-1}}(P_{k-2})), \ 2 \leq k \leq n \ ,
\end{align*}
where $\Sigma_{\pi^{k}_{k-1}}(P_{k-2})\defeq \Sigma_{\pi^{k}_{k-1}}(\pi_k(P_{k-2}),\pi_{k-1}(P_{k-2}))$.

By definition of the fiber polytope, we have $P_k \subset \pi_k^{-1}(r_k)$, where $r_k$ is the barycenter of $\pi_k(P)$, and $\dim P_k = \dim P - \min\{ \dim P,k\}$ for all $0\leq k \leq n$.
In the case where $P$ is full dimensional, we have $\dim P_k = n-k$, and all the $P_k, 1 \leq k \leq n$ are made out of a projection to a 1-dimensional polytope, so they are monotone path polytopes.
Note that flags of polytopes and their associated iterated fiber polytopes were studied in \cite{BilleraSturmfels94}.
Note also that in this definition, we make repeated use of the following general fact: given two consecutive projections $P \xra{\pi} Q \to R$, the fiber polytope $\Sigma(Q,R)$ is the image of the fiber polytope $\Sigma(P,R)$ under the projection $\pi$ \cite[Lemma 2.3]{BilleraSturmfels92}.

\anibal{Connection with toric variety?}

% \subsection{Zonotopes} \label{ss:zonotopes}

%Keep for our next project!

% If $P$ is a zonotope, then all the $P_k$ are also zonotopes.
% Indeed, in this case $P$ is the projection of a cube $C \xra{\pi} P$.
% \cite[Theorem 4.1]{BilleraSturmfels92} says that the fiber polytope $\Sigma(C,P)$ of this projection is itself a zonotope, the \emph{fiber zonotope} of $P$.
% We consider the flag $C \xra{\pi} P \xra{\pi_1} I$, where $I$ is the $1$-dimensional image of $P$ under the projection $\pi_1$.
% Using the functoriality property \cite[Lemma 2.3]{BilleraSturmfels92}, we have that $\Sigma(P,I)=\pi(\Sigma(C,I))$ is the projection of a zonotope, thus itself a zonotope.
% Repeating this argument, we get that all the $P_k$ are zonotopes.
% \anibal{What is a zonotope? Why are they interesting?}

\subsection{Centrally symmetric polytopes} \label{ss:centrally-symmetric}

A polytope $P$ is \emph{centrally symmetric} if there is a point $r$ in $P$ such that for every point $p \in P$, we have $r-p \in P$.

\begin{lemma}
	\label{l:centrally-symmetric}
	If $P$ is centrally symmetric, the iterated monotone path sequence is made up of centrally symmetric polytopes. 
\end{lemma}

\begin{proof}
	Every centrally symmetric polytope $P$ is the projection of a cross-polytope $T \xra{\pi} P$.
	The fiber polytope $\Sigma(T,P)$ of this projection is itself centrally symmetric \cite[Theorem 5.1]{BilleraSturmfels92}.
	We consider the flag $T \xra{\pi} P \xra{\pi_1} I$, where $I$ is the $1$-dimensional image of $P$ under the projection $\pi_1$.
	Using the functoriality property \cite[Lemma 2.3]{BilleraSturmfels92}, we have that $\Sigma(P,I)=\pi(\Sigma(T,I))$ is the projection of a centrally symmetric polytope, thus itself centrally symmetric.
	Repeating this argument, we get that all the $P_k$ are centrally symmetric.
\end{proof}

