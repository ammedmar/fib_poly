% !TEX root = ../fib_poly.tex

\subsection{Polytopes}

A polytope $P \subset \mathbb{R}^n$ is the convex hull of a finite set of points. Equivalently, it is defined as the bounded intersection of a finite number of half-spaces. 

Let $\pi \colon P \to Q$ be a projection of polytopes, that is, an affine map $\pi \colon \R^p \to \R^q$ such that $\pi(P)=Q$ for two polytopes $P\subset \R^p$ and $Q \subset \R^q$.
One can associate to it the \emph{fiber polytope} \[\Sigma(P,Q) \defeq \left\{ \frac{1}{vol(Q)}\int_Q \gamma(x)dx \ \colon \ \gamma \text{ is a section of }\pi \right\} \ , \] whose face lattice is isomorphic to the lattice of $\pi$-coherent subdivisions of $Q$ \cite{BilleraSturmfels92}.
See also \cite[Chapter 9]{Ziegler95} for a detailed account.
The vertices of $\Sigma(P,Q)$ correspond to tight coherent subdivisions, that is, subdivisions $\pi(\mathcal{F})$ where for each face $F \in \mathcal{F}\subset\mathcal{L}(P)$ we have $\dim(\pi(F))=\dim(F)$.
Any linear function $c\in (\R^p)^{*}$ induces a subdivision $\pi(\mathcal{F}^c)$.
When this subdivision is tight, there is a unique section $\gamma^c$ of $\pi$ which maximizes $c$ in each fiber.

Morover, the fiber polytope $\Sigma(P,Q)$ lives in the preimage of the barycenter $r$ of $Q$, that is, $\Sigma(P,Q)\subset\pi^{-1}(r)\cap P$.
