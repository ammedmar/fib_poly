% !TEX root = ../fib_poly.tex

\section{Fiber polytopes and Steenrod diagonals} \label{s:fiber polytopes and Steenrod diagonals}

Consider the diagonal embedding
\[
\begin{tikzcd}[row sep=0, column sep=small]
\Delta \colon \R^n \arrow{r} & \R^n \times \R^n \\
\phantom{\Delta \colon} z \arrow[r, |->] & (z,z).
\end{tikzcd}
\]
Denote by $\{e_j\}$ the standard basis of $\R^n$. 
Then, $\{\Delta (e_j)\}$ is a basis of $\Ima \Delta$ and we have an isomorphism $\R^n \cong \Ima \Delta$. 
A basis for the orthogonal complement $\Ima \Delta^{\bot}$ is $\{(e_j,-e_j)\}$, and we have an isomorphism $\R^n \cong \Ima \Delta^{\bot}$. 
We consider the projection 
\[
\begin{tikzcd}[row sep=0, column sep=small]
\pi \colon \R^n \times \R^n \arrow{r} & \R^n \\
\phantom{\pi_0 \colon} (x,y) \arrow[r, |->] & \frac{(x+y)}{2}.
\end{tikzcd}
\]
For any $z \in \R^n$, we have $\pi^{-1}(z)=\Delta(z)+\ker \pi$, and an isomorphism
\begin{equation*}
\begin{matrix}
 \iota_z & : & \R^n  & \cong & \pi^{-1}(z) \\
 & & v  & \mapsto & \Delta(z)+(v,-v).
\end{matrix}
\end{equation*}
We can see this isomorphism as the discrete analogue of the isomorphism between the tangent bundle of a manifold $M$ and the normal bundle of the diagonal submanifold of $M\times M$. 

Let $P \subset \R^n$ be a polytope. We consider the projection
\[
\begin{tikzcd}[row sep=0, column sep=small]
\pi_0 \colon P \times P \arrow{r} & P \\
\phantom{\pi_0 \colon} (x,y) \arrow[r, |->] & \frac{(x+y)}{2}.
\end{tikzcd}
\]
In this case, the section $\triangle_0$ associated to any vertex $s_0$ of the fiber polytope $P_0 \defeq \Sigma(P \times P,P)$ is a cellular approximation of the diagonal of $P$ \cite[Proposition 5]{MTTV19}. 
We call it the \emph{polytope of diagonals} of $P$.
We have $P_0 \subset \pi_0^{-1}(r)$, where $r$ is the centroid of $P$.

The product $P \times P$ admits a natural action of the symmetric group $\Sym_2$ by permutation of the factors. Let $\sigma$ denote the non-trivial permutation $(x,y)\mapsto (y,x)$.

\begin{proposition}
	Let $P$ be a polytope, and let $r$ be its centroid. The polytope of diagonals $\Sigma (P \times P, P)$ is centrally symmetric with respect to the point $(r,r)$. Moreover, the symmetry is given by permutation of the factors. 
\end{proposition}
\begin{proof} 
    Observe first that any fiber $\pi_0^{-1}(z)$, for $z \in P$, is invariant under the action of $\sigma$.  
    Using the description of $\pi_0^{-1}(z)$ given by the isomorphism $\iota_z$, one can see directly that this action defines a central symmetry with respect to the point $\Delta (z)$.
    The fiber polytope $P_0$ being a Minkowski sum of centrally symmetric polytopes \cite[Theorem 1.5]{BilleraSturmfels92}, it is also centrally symmetric.
\end{proof}

\begin{corollary}
	Any tight coherent section $\triangle_0$ of $\pi_0$ has an opposite section $\triangle_0^{\op}$, given by $\triangle_0^{\op}(z)=\sigma\triangle_0(z)$.
\end{corollary}

\begin{proof}
	The normal fan $\mathcal{N}_{P_0}$ of $P_0$ is a fan of the space $(\ker \pi)^{*}$ \cite[Section 2]{BilleraSturmfels94}. 
    This can be seen to be the space of linear functionals $\phi_v : \R^n \times \R^n \to \R$ of the form $\phi_v(x,y) \defeq \langle x-y, v \rangle$, where $v \in \R^n$, via the canonical isomorphism $\{(v,-v) \ | \ v \in \R^n \}=\ker \pi \cong (\ker \pi)^{*}=\{\langle -, (v,-v)\rangle \ | \ v \in \R^n \}$. 
    Now let $\phi_v$ be in the interior of a maximal cone of $\mathcal{N}_{P_0}$.
	For any $z\in P$, the point $\triangle_0(z)$ maximizes $\phi_v$ in $\pi_0^{-1}(z)$.
	Then, the point $\triangle_0^{\op}(z) \defeq \sigma \triangle_0(z)$ maximizes $\phi_{-v}$ in $\pi_0^{-1}(z)$.
\end{proof}

Now choose a tight coherent section $\triangle_0$.
For $n\geq 0$, let $I^n=[0,1]^n$ denote the $n$-dimensional cube.
We want to build two families of cellular maps
\begin{eqnarray*}
\triangle_n, \triangle^{\op}_n & : & I^n \times P \to P \times P  \ ,
\end{eqnarray*}
which are equivariant under the action of $\Sym_2$ sending $\triangle_n$ to $\triangle^{\op}_n$, and such that $\triangle_i$ and $\triangle_i^{\op}$ are homotopies between $\triangle_{i-1}$ and $\triangle_{i-1}^{\op}$ for all $i \geq 1$.

Let $s_0$ be the vertex of $P_0$ associated to $\triangle_0$, and denote by $s_0^{\op}$ its opposite vertex, associated to $\triangle_0^{\op}$. We set $v_0 \defeq s_0^{\op}-s_0$. 

Consider now the linear function $\langle - , v_0 \rangle$.
It defines a projection of polytopes $\pi_1 \colon P_0 \to I$ via the formula
\[
x \mapsto \frac{\langle x-s_0, v_0 \rangle}{||v_0||^2}.
\]
We consider the fiber polytope $P_1 \defeq \Sigma(P_0,I)$ associated to this projection, also called the \emph{monotone path polytope} \cite[Chapter 9]{Ziegler95}.

Choose a vertex $s_1$ in $P_1$.
It defines a monotone path between $s_0$ and $s^{\op}_0$ in $P_0$, that is, a sequence of vertices and edges $s_0 = s_0^1, e_0^1, s_0^2, e_0^2, \dots, e_0^k, s_0^{k+1} = s^{\op}_0$ on which the linear function $\langle - , v_0 \rangle$ takes increasing values.
These vertices and edges define a sequence of subdivisions of $P_0$ that interpolate between the two cellular approximations of the diagonal $\triangle_0$ and $\triangle^{\op}_0$.
Let us denote the tight coherent sections associated to $s_0^i$ by $\triangle_0^i$, for $1 \leq i \leq k+1$.

The vertex $s_1$ of $P_1$ defines a tight coherent subdivision of $I$.
We denote the vertices of this subdivision by $x_i \defeq \pi_1(s_0^i)$, for $1 \leq i \leq k+1$.
We define a map
\[
\triangle_1 \colon I \times P \to P \times P,
\]
on the segment $[x_i, x_{i+1}]$, $1 \leq i \leq k$, by the formula
\[
(t,z) \mapsto \left( \frac{x_{i+1}-t}{x_{i+1}-x_i} \right) \triangle_0^{i}(z) + \left( \frac{t-x_{i}}{x_{i+1}-x_i} \right) \triangle_0^{i+1}(z).
\]
We obtain a cellular, continuous and piecewise-linear map.
Notice that, by construction, we have $\triangle_1(0,z)=\triangle_0(z)$ and $\triangle_1(1,z)=\triangle_0^{\op}(z)$, so $\triangle_1$ is an homotopy between $\triangle_0$ and $\triangle_0^{\op}$.

Since the normal fan of $P_0$ is centrally symmetric, the vertex $s_1$ of $P_1$ has an opposite vertex $s_1^{\op}$ in $P_1$: it is associated to the monotone path from $s_0$ to $s_0^{\op}$ in $P_0$ given by taking the opposite of each face in the path from $s_0$ to $s^{\op}_0$, in the reverse order.
With the notations above, it is given by $(s_0^{k+1})^{\op},(e_0^k)^{\op},\ldots,(e_0^1)^{\op},(s_0^1)^{\op}$.
In the same fashion as for $\triangle_1$, this path defines a map
\[
\triangle_1^{\op} \colon I \times P \to P \times P.
\]
Notice that $I$ is endowed here with a different tight coherent subdivision than for $\triangle_1$.
Again, $\triangle_1^{\op}$ is an homotopy between $\triangle_0$ and $\triangle_0^{\op}$.

Set $v_1 \defeq s_1^{\op}-s_1$, and consider the linear function $\langle - , v_1 \rangle$.
It defines a projection of polytopes $\pi_2 \colon P_1 \to I$ via the formula
\[
x \mapsto \frac{\langle x-s_1, v_1 \rangle}{||v_1||^2}.
\]
We consider the fiber polytope $P_2 \defeq \Sigma(P_1,I)$ associated to this projection.

Choose a vertex $s_2$ in $P_2$. It defines a monotone path $s_1 = s_1^1, e_1^1, s_1^2, e_1^2, \dots, e_1^l, s_1^{l+1} = s^{\op}_1$.
These vertices and edges define a sequence of subdivisions of $I$ that interpolate between the subdivisions associated to $s_1$ and $s_1^{\op}$.
Let us denote the tight coherent section associated to $s_1^i$ by $\triangle_1^i$, for $1\leq i \leq l+1$.

The vertex $s_2$ of $P_2$ defines a tight coherent subdivision of $I$.
We denote the vertices of this subdivision by $y_i \defeq \pi_2(s_1^i)$, for $1 \leq i \leq l+1$.
We define a map
\[
\triangle_2 \colon I \times (I \times P) \to P \times P,
\]
on the segment $[y_i,y_{i+1}]$, $1 \leq i \leq l$, by the formula
\[
(t,z) \mapsto \left(\frac{y_{i+1}-t}{y_{i+1}-y_i}\right)\triangle_1^{i}(z)+\left(\frac{t-y_{i}}{y_{i+1}-y_i}\right)\triangle_1^{i+1}(z).
\]
We obtain a cellular, continuous and piecewise-linear map.
Notice that, by construction, we have $\triangle_2(0,z)=\triangle_1(z)$ and $\triangle_2(1,z)=\triangle_1^{\op}(z)$, so $\triangle_2$ is an homotopy between $\triangle_1$ and $\triangle_1^{\op}$.

Now we iterate the procedure, define $P_3$ and choose a vertex $s_3$, and so on.
This process will stop at step $\dim P$.
Indeed, we have
\begin{eqnarray*}
    \dim P_0 & = & \dim \Sigma P = 2 \dim P - \dim P = \dim P \\
    \dim P_1 & = & \dim P_0 -1 = \dim P -1 \\
    \dim P_2 & = & \dim P_1 - 1 = \dim P -2 \\
    \vdots & & \vdots  \\
    \dim P_n & = &\dim P - n
\end{eqnarray*}
So if $P$ has dimension $n$, then $P_n$ has dimension 0 and there is no more fiber polytope to be constructed.

We shall call \emph{vertex system} a choice of vertices $s_i$ in the fiber polytopes $P_i$, for $i\geq 0$. In summary, we have proven the following result.

\begin{theorem}
    A vertex system defines a Steenrod diagonal of $P$.
\end{theorem}

The preceding construction exhibits a natural family of vectors. 

\begin{theorem}
    A vertex system defines an orthogonal basis of $\LinSpan(P)$.
\end{theorem}

\begin{proof}
    Any choice of vertices $s_i$ in the fiber polytopes $P_i$ gives a family of vectors $v_i \defeq s_i^{op} - s_i$, which define the successive projections $\pi_i$. Let us define $T_0 \defeq \pi_0^{-1}(r)$, where $r$ is the centroid of $P$. For $i \geq 1$, let us define $T_i \defeq \LinSpan(\pi_{i}^{-1}(1/2))$. We have $P_i \subset T_i$ for each $i \geq 0$, by definition of the fiber polytope. 
    
    Since the projection $\pi_i$ is defined by taking the scalar product with $v_{i-1}$, the linear space $T_i$ is an hyperplane of $T_{i-1}$, orthogonal to the vector $v_{i-1}$. Also, we have the inclusion $T_i \subset T_{i-1}$ for all $i\geq 1$, so the vectors $v_i$, $i \geq 0$, are all orthogonal to each other. Since there are exactly $n$ such non-zero vectors, they form an orthogonal basis of $T_0$. 
    
    The space $T_0$ is, modulo translation by $(r,r)$, a linear subspace of $\ker \pi = \{(v,-v) \ | \ v \in \R^n \}$, which has a canonical identification with $\R^n$. This isomorphism $(v,-v) \mapsto v$ preserves orthogonality, thus the orthogonal basis of $T_0$ gives an orthogonal basis of $\LinSpan(P) \subset \R^n$. [Restrict ourselves to a full-dimensional polytope for simplicity??]
\end{proof}


\begin{question}
    Given a full-dimensional polytope $P \subset \R^n$, which basis of $\R^n$ define a Steenrod diagonal of $P$?
\end{question}

\begin{example}[The simplices]
\end{example}

\begin{example}[Zonotopes]
\end{example}





