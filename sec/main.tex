% !TEX root = ../fib_poly.tex

\section{Fiber polytopes and Steenrod diagonals} \label{s:fiber polytopes and Steenrod diagonals}

Let $\pi \colon P \to Q$ be a projection of polytopes, that is, an affine map $\pi \colon \R^p \to \R^q$ such that $\pi(P)=Q$ for two polytopes $P\subset \R^p$ and $Q \subset \R^q$.
One can associate to it the \emph{fiber polytope} \[\Sigma(P,Q) \defeq \left\{ \frac{1}{vol(Q)}\int_Q \gamma(x)dx \ \colon \ \gamma \text{ is a section of }\pi \right\} \ , \] whose face lattice is isomorphic to the lattice of $\pi$-coherent subdivisions of $Q$ \cite{BilleraSturmfels92}.
See also \cite[Chapter 9]{Ziegler95} for a detailed account.
The vertices of $\Sigma(P,Q)$ correspond to tight coherent subdivisions, that is, subdivisions $\pi(\mathcal{F})$ where for each face $F \in \mathcal{F}\subset\mathcal{L}(P)$ we have $\dim(\pi(F))=\dim(F)$.
Any linear function $c\in (\R^p)^{*}$ induces a subdivision $\pi(\mathcal{F}^c)$.
When this subdivision is tight, there is a unique section $\gamma^c$ of $\pi$ which maximizes $c$ in each fiber.

Morover, the fiber polytope $\Sigma(P,Q)$ lives in the preimage of the barycenter $r$ of $Q$, that is, $\Sigma(P,Q)\subset\pi^{-1}(r)\cap P$.

Let $P\subset\R^n$ be a polytope.
Following \cite{MTTV19} and \cite{GLA21}, we consider the projection $\pi \colon P\times P \to P$, $(x,y)\mapsto (x+y)/2$.
In this case, the section $\triangle$ associated to any vertex of the fiber polytope $\Sigma P \defeq \Sigma(P\times P,P)$ is a cellular approximation of the diagonal of $P$.

Observe that any fiber $\pi^{-1}(z), z \in P$ is contained in an hyperplane $H$ in $\R^n\times\R^n$ for which $(1,\ldots,1)$ is a normal vector.
In particular, this is true for the fiber of the barycenter $r$ of $P$.
So, the fiber polytope $\Sigma P$ lies in an hyperplane with normal vector $(1,\ldots,1)$.
This means that if we want to study the normal fan of $\Sigma P$, it is enough to restrict our attention to linear functions $c=(c_1,c_2) \in (\R^n \times \R^n)^{*}\cong (\R^n)^{*}\times (\R^n)^{*}$ for which $c_1=-c_2$.

The product $P\times P$ admits a natural action of the symmetric group $\S_2$ by permutation of the factors.
Let $\sigma_2$ denote the non-trivial permutation $(x,y)\mapsto (y,x)$.

\begin{proposition}
	Any tight coherent section $\triangle$ has an opposite section $\triangle^{\op}$, given by $\triangle^{\op}(z)=\triangle(z)\sigma_2$.
\end{proposition}

\begin{proof}
	Let $c\in (\R^n\times\R^n)^{*}$ be in the interior of a maximal cone of $\mathcal{N}_{\Sigma P}$.
	By the preceding observation, we can restrict our attention to a linear function $c$ of the form $c=(c_1,-c_1)$.
	For any $z\in P$, the point $\triangle(z)$ in $P\times P$ maximizes $c$ in $\pi^{-1}(z)$.
	Then, the point $\triangle^{\op}(z) \defeq \triangle(z)\sigma_2$ maximizes $-c$ in $\pi^{-1}(z)$.
\end{proof}

This means that the normal fan of $\Sigma P$ is centrally symmetric: any face $F$ of $\Sigma P$ has an "opposite" face $F^{\op}$, defined by $\mathcal{N}_{\Sigma P}(F^{\op})=-\mathcal{N}_{\Sigma P}(F)$.

Now choose a tight coherent section $\triangle_0\defeq\triangle$.
For $n\geq 0$, let $I^n=[0,1]^n$ denote the $n$-dimensional cube.
We want to build two families of cellular maps
\begin{eqnarray*}
\triangle_n, \triangle^{\op}_n & : & I^n \times P \to P\times P  \ ,
\end{eqnarray*}
which are equivariant under the action of $\S_2$ sending $\triangle_n$ to $\triangle^{\op}_n$.

We already constructed $\triangle_0$ and $\triangle^{\op}_0$; this amounted to the choice of a linear function $c_0$ in an inclusion-maximal cone of the normal fan $\mathcal{N}_{P_0}$ of the fiber polytope $P_0\defeq\Sigma P$, or equivalently to the choice of a vertex $v_0$ of $P_0$.
The section $\triangle^{\op}_0$ is associated with the linear function $-c_0$, or equivalently to another vertex $v^{\op}_0$ of $P_0$, the opposite vertex of $v_0$.

Consider now the linear function $\langle - , v^{\op}_0-v_0 \rangle$.
It defines a projection of polytopes $\pi_0 \colon P_0 \to I$ via the formula
\[
x \mapsto \frac{x-\langle v_0, v^{\op}_0-v_0 \rangle}{||v^{\op}_0-v_0||^2}.
\]
We consider the fiber polytope $P_1 \defeq \Sigma(P_0,I)$ associated to this projection, also called the \emph{monotone path polytope} \cite[Chapter 9]{Ziegler95}.
Choose a vertex $v_1$ in $P_1$.
It defines a monotone path between $v_0$ and $v^{\op}_0$ in $P_0$, that is, a sequence of vertices and edges $v_0 = v_0^1, e_0^1, v_0^2, e_0^2, \dots, e_0^k, v_0^{k+1} = v^{\op}_0$ on which the linear function $\langle - , v^{\op}_0-v_0 \rangle$ takes increasing values.
These vertices and edges define a sequence of subdivisions of $P$ that interpolate between the two cellular approximations of the diagonal $\triangle_0$ and $\triangle^{\op}_0$ of $P$.
Let us denote the tight coherent sections associated to $v_0^i$'s by $\triangle_0^i$, for $1 \leq i \leq k+1$.

The vertex $v_1$ of $P_1$ defines a tight coherent subdivision of $I$.
We denote the vertices of this subdivision by $x_i \defeq \pi_0(v_0^i)$, for $1 \leq i \leq k+1$.
We define a map
\[
\triangle_1 \colon I \times P \to P \times P,
\]
on the segment $[x_i, x_{i+1}]$, $1 \leq i \leq k$, by the formula
\[
(t,z) \mapsto \left( \frac{x_{i+1}-t}{x_{i+1}-x_i} \right) \triangle_0^{i}(z) + \left( \frac{t-x_{i}}{x_{i+1}-x_i} \right) \triangle_0^{i+1}(z).
\]
We obtain a cellular, continuous and piecewise-linear map.
Notice that, by construction, we have $\triangle_1(0,z)=\triangle_0(z)$ and $\triangle_1(1,z)=\triangle_0^{\op}(z)$, so $\triangle_1$ is an homotopy between $\triangle_0$ and $\triangle_0^{\op}$.

Since the normal fan of $\Sigma P$ is centrally symmetric, the vertex $v_1$ of $P_1$ has an opposite vertex $v_1^{\op}$ in $P_1$: it is associated to the monotone path from $v_0$ to $v_0^{\op}$ in $P_0$ given by taking the opposite of each face in the path from $v_0$ to $v^{\op}_0$, in the reverse order.
With the notations above, it is given by $(v_0^{k+1})^{\op},(e_0^k)^{\op},\ldots,(e_0^1)^{\op},(v_0^1)^{\op}$.
In the same fashion as for $\triangle_1$, the this path defines a map
\[
\triangle_1^{\op} \colon I \times P \to P \times P.
\]
Notice that $I$ is endowed here with a different tight coherent subdivision than for $\triangle_1$.
Again, $\triangle_1^{\op}$ is an homotopy between $\triangle_0$ and $\triangle_0^{\op}$.

Consider now the linear function $\langle - , v^{\op}_1-v_1 \rangle$.
It defines a projection of polytopes $\pi_1 \colon P_1 \to I$ via the formula
\[
x \mapsto \frac{x-\langle v_1, v^{\op}_1-v_1 \rangle}{||v^{\op}_1-v_1||^2}.
\]
We consider the fiber polytope $P_2 \defeq \Sigma(P_1,I)$ associated to this projection.
Choose a vertex $v_2$ in $P_2$.

It defines a monotone path $v_1 = v_1^1, e_1^1, v_1^2, e_1^2, \dots, e_1^l, v_1^{l+1} = v^{\op}_1$.
These vertices and edges define a sequence of subdivisions of $I$ that interpolate between the subdivisions associated to $v_1$ and $v_1^{\op}$.
Let us denote the tight coherent sections associated to the $v_1^i$'s by $\triangle_1^i$, for $1\leq i \leq l+1$.

The vertex $v_2$ of $P_2$ defines a tight coherent subdivision of $I$.
We denote the vertices of this subdivision by $y_i \defeq \pi_1(v_1^i)$, for $1 \leq i \leq l+1$.
We define a map
\[
\triangle_2 \colon I \times (I \times P) \to P \times P,
\]
on the segment $[y_i,y_{i+1}]$, $1 \leq i \leq l$, by the formula
\[
(t,z) \mapsto \left(\frac{y_{i+1}-t}{y_{i+1}-y_i}\right)\triangle_1^{i}(z)+\left(\frac{t-y_{i}}{y_{i+1}-y_i}\right)\triangle_1^{i+1}(z).
\]
We obtain a cellular, continuous and piecewise-linear map.
Notice that, by construction, we have $\triangle_2(0,z)=\triangle_1(z)$ and $\triangle_2(1,z)=\triangle_1^{\op}(z)$, so $\triangle_2$ is an homotopy between $\triangle_1$ and $\triangle_1^{\op}$.

Now we iterate the procedure, define $P_3$ and choose a vertex $v_3$, and so on.
This process will stop at step $\dim P$.
Indeed, we have
\begin{eqnarray*}
    \dim P_0 & = & \dim \Sigma P = 2 \dim P - \dim P = \dim P \\
    \dim P_1 & = & \dim P_0 -1 = \dim P -1 \\
    \dim P_2 & = & \dim P_1 - 1 = \dim P -2 \\
    \vdots & & \vdots  \\
    \dim P_n & = &\dim P - n
\end{eqnarray*}
So if $P$ has dimension $n$, then $P_n$ has dimension 0 and there is no more fiber polytope to be constructed.

So, given a family of vertices $v_i$ in the fiber polytopes $P_i$, $i\geq 0$, we have constructed a tower of homotopies, equivariant under the action of the symmetric groups.

\begin{example}[The simplices]
\end{example}

\begin{example}[The cubes]
\end{example}

\begin{example}[The associahedra]
\end{example}

\begin{example} What happens is $P$ is a zonotope? Do we get a sequence of zonotopes?
\end{example}





